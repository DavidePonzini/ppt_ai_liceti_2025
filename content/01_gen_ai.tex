\makesectionframe{Introduzione}

\begin{contentframe}
    \frametitle{Cosa sono le intelligenze artificiali generative?}

    \begin{itemize}
        \item Le intelligenze artificiali \textit{(IA o AI)} generative sono strumenti in grado di generare testo, immagini o altro in risposta ad una richiesta

        \bigskip
        \item Tra le più famose: ChatGPT, Gemini, DeepSeek, Copilot, Claude, Llama, Mistral, ...

        \bigskip
        \item Noi facciamo una domanda e lo strumento genera una risposta
        \begin{itemize}
            \item Vediamo un esempio...
            \item \url{https://chatgpt.com/}
        \end{itemize}
    \end{itemize}
\end{contentframe}

\makesectionframe{Le intelligenze artificiali}

\begin{contentframe}
    \frametitle{IA VS IA Generative}

    \begin{itemize}
        \item Introduciamo una distinzione importante...
    \end{itemize}

    \bigskip
    \begin{block}{Intelligenza artificiale}
        Programma che svolge un compito che richiede normalmente ``intelligenza umana''

        \small\textit{In genere operazioni per cui serve ragionare}
    \end{block}

    \begin{block}{Intelligenza artificiale generativa}
        Intelligenza artificiale specializzata nella generazione di contenuti
        
        \small\textit{Esempio: testo, immagini, ...}
    \end{block}
\end{contentframe}


\begin{exampleframe}
    \frametitle{Le intelligenze artificiali}
    \framesubtitle{Esempi}

    \begin{itemize}
        \item Mostrare ad un utente la pubblicità più rilevante, conoscendo i suoi interessi

        \bigskip
        \item Proporre ad un utente una nuova canzone da ascoltare, a partire dalla playlist corrente

        \bigskip
        \item Capire se la nostra auto deve svoltare, andare dritta o fermarsi, utilizzando i dati provenienti dai vari sensori
    \end{itemize}
\end{exampleframe}

\begin{exampleframe}
    \frametitle{Le intelligenze artificiali generative}
    \framesubtitle{Esempi}

    \begin{itemize}
        \item Tradurre efficientemente da una lingua ad un'altra

        \bigskip
        \item Scrivere un testo in un certo stile o secondo determinate regole

        \bigskip
        \item Dialogare \textit{(o meglio, simulare un dialogo)} con una persona
    \end{itemize}
\end{exampleframe}

\begin{contentframe}
    \frametitle{AI deduttiva vs AI induttiva}

    \begin{itemize}
        \item Come fanno le AI a capire qual è la risposta giusta ad un problema?

        \bigskip
        \item Due approcci:
        \begin{itemize}
            \item AI deduttiva (o simbolica)
            \item AI induttiva
        \end{itemize}
    \end{itemize}
\end{contentframe}

\begin{contentframe}
    \frametitle{AI deduttiva vs AI induttiva}

    \begin{block}{AI deduttiva (o simbolica)}
        \textbf{Imparare dalle regole.}
        Rappresenta i problemi e cerca di risolverli tramite una serie di regole logiche.
        \\\textbf{Limitazione:} esistono infiniti problemi non risolvibili con questo approccio.
    \end{block}

    \begin{block}{AI induttiva}
        \textbf{Imparare dall'esperienza.}
        Estrae regole e modelli a partire da grandi quantità di dati.
        \\\textbf{Limitazione:} il modello è vincolato ai dati su cui è allenato.
    \end{block}
\end{contentframe}

\begin{contentframe}
    \frametitle{AI deduttiva vs AI induttiva}
    \framesubtitle{Esempio}

    \begin{itemize}
        \item Supponiamo di voler predire il voto che un certo studente prenderà all'interrogazione, sapendo il numero di ore che ha studiato

        \pause
        \bigskip
        \item \textbf{AI deduttiva:} partendo la problema, si cercano delle regole logiche che permettano di calcolare i voti
        \pause
        \item \textbf{AI induttiva:}  partendo da tanti voti di studenti, si cerca di trovare un calcolo statistico abbastanza accurato 
    \end{itemize}
\end{contentframe}

\begin{exerciseframe}
    \frametitle{AI deduttiva}

    \begin{itemize}
        \item Provare a definire un insieme di regole per indovinare che voto prenderà un certo studente all'interrogazione. Esempio:
    \end{itemize}

    \centering
    \smartarttree[
            Lo studente ha studiato?
            [Sì[??][??]]
            [No[??][??]]
        ]{}
\end{exerciseframe}

\begin{contentframe}
    \frametitle{Machine Learning}

    \begin{itemize}
        \item L'AI induttiva si basa sul ``Machine Learning''
    \end{itemize}

    \bigskip
    \begin{block}{Machine Learning}
        Procedimento statistico per predire con una certa probabilità la risposta corretta
        \begin{itemize}
            \item Ha bisogno di dati da cui partire
        \end{itemize}
    \end{block}
\end{contentframe}

\begin{contentframe}
    \frametitle{Machine Learning}
    \framesubtitle{KNN}

    \begin{itemize}
        \item Vediamo un esempio semplificato di Machine Learning
        \begin{itemize}
            \item KNN \textit{(``K-Nearest Neighbours'')}
        \end{itemize}

        \bigskip
        \item Sapendo il numero di ore che uno studente ha studiato, vogliamo predire il voto che prenderà all'interrogazione

        \bigskip
        \item Abbiamo prima di tutto bisogno di sapere le ore di studio e i voti di altre persone
    \end{itemize}
\end{contentframe}

\begin{exampleframe}
    \frametitle{Machine Learning}
    \framesubtitle{KNN}

    \begin{itemize}
        \item Che voto prenderà uno studente che ha studiato\\esattamente 1 ora?
        \item<2-> Guardiamo i voti più vicini e facciamo la media (\only<2-4>{$k = 1$}\only<5->{$k = 3$})
    \end{itemize}

    \only<1-2>{
        \centering
        \image*[1][.4]{knn-scatterplot1.pdf}
    }%
    \only<3>{
        \centering
        \image*[1][.4]{knn-scatterplot2.pdf}
    }%
    \only<4>{
        \centering
        \image*[1][.4]{knn-scatterplot3.pdf}
    }%
    \only<5>{
        \centering
        \image*[1][.4]{knn-scatterplot4.pdf}
    }%
    \only<6>{
        \centering
        \image*[1][.4]{knn-scatterplot5.pdf}
    }%

    \begin{itemize}
        \item<4-> Lo studente probabilmente prenderà \only<4-5>{$6.0$}\only<6>{$6.8$}
    \end{itemize}
\end{exampleframe}

\begin{contentframe}
    \frametitle{Machine Learning}

    \begin{itemize}
        \item Dall'esempio di prima emerge un problema

        \bigskip
        \item Qual è il valore di $k$ che fornisce risultati migliori?
        \begin{itemize}
            \item $k$ troppo piccolo dà risultati poco precisi
            \item $k$ troppo grande dà sempre lo stesso risultato per tutti
        \end{itemize}

        \pause
        \begin{block}{Training}
            Si provano tutti i valori dei parametri \textit{(in questo caso solo $k$)} fino a trovare quello che fornisce risultati più accurati
        \end{block}

        \begin{itemize}
            \item Per un solo parametro è facile, ma le AI generative ne hanno miliardi!
        \end{itemize}
    \end{itemize}
\end{contentframe}

\begin{contentframe}
    \frametitle{Machine Learning}
    \framesubtitle{Training}

    \begin{itemize}
        \item Fase iniziale con cui si trovano i parametri migliori
        \begin{itemize}
            \item Possono variare in base ai dati o allo scopo
        \end{itemize}

        \bigskip
        \item Una volta trovati i parametri ``ideali'', il modello può essere usato
        \item Più dati abbiamo, più i parametri scelti in fase di training saranno efficaci
        
        \begin{itemize}
            \item ChatGPT è stato allenato su più di 500 GB di dati
        \end{itemize}
    \end{itemize}
\end{contentframe}

\begin{exampleframe}
    \frametitle{Training}

    \begin{itemize}
        \item Vogliamo preparare la pasta al sugo
        \item Abbiamo a disposizione un esperto a cui fare assaggiare tutti i piatti che vogliamo
    \end{itemize}

    \bigskip
    \centering
    \image*[.45]{pasta.jpg}
\end{exampleframe}

\begin{exampleframe}
    \frametitle{Training}
    \begin{tikzimage}
        \node[draw, ellipse] (acqua) at (0, 3) {Acqua};
        \node[draw, ellipse] (pasta) at (0, 2) {Pasta};
        \node[draw, ellipse] (salsa) at (0, 1) {Salsa di pomodoro};
        \node[draw, ellipse] (olio) at (0, 0) {Olio};
        \node[draw, ellipse] (grana) at (0, -1) {Grana};
        \node[draw, ellipse] (sale) at (0, -2) {Sale};

        \node[draw, circle] (target) at (8, 0) {\faThumbsUp~\faThumbsDown};
        
        \draw[->] (acqua) -- node[above] {peso} (target);
        \draw[->] (pasta) -- node[above] {peso} (target);
        \draw[->] (salsa) -- node[above] {peso} (target);
        \draw[->] (olio) -- node[above] {peso} (target);
        \draw[->] (grana) -- node[above] {peso} (target);
        \draw[->] (sale) -- node[above] {peso} (target);
    \end{tikzimage}
\end{exampleframe}

\begin{exampleframe}
    \frametitle{Training}
    \framesubtitle{Ridurre l'errore di apprendimento}

    \begin{itemize}
        \item Algoritmo basato sul principio del ``perdere quota ad ogni passo''
    \end{itemize}

    \bigskip
    \begin{columns}
        \col{.5}
        \centering
        \image*{hiking.jpg}
        
        \col{.5}
        \centering
        \begin{tikzimage}
            % Convex function (blue line)
            \draw[thick, domain=0.5:3.5, smooth, variable=\x, blue] 
                plot ({\x}, {0.5*(\x - 2.3)^2});
            \fill[blue] (1, {0.5*(1 - 2.3)^2}) circle (2pt);
        
            % Non-convex function (red line)
            \draw[thick, domain=0.5:3.5, smooth, variable=\x, red] 
                plot ({\x}, {0.5*sin(3.6*deg(\x)) - 0.5*\x - 0.5});
            \fill[red] (1, {0.5*sin(3.6*deg(1)) - 0.5*1 - 0.5}) circle (2pt);
        \end{tikzimage}
    \end{columns}
\end{exampleframe}

\makesectionframe{Funzionamento AI generativa}

\begin{contentframe}
    \frametitle{Come funzionano le AI generative?}

    \begin{itemize}
        \item Sistema complesso in grado di estrarre il significato da una frase e di elaborarlo
        \begin{itemize}
            \item Calcoli matematici complessi permettono di capire quali informazioni sono più rilevanti
        \end{itemize}
        
        \bigskip
        \item Dopo aver elaborato la risposta, la converte in testo (o altro formato)
    \end{itemize}
\end{contentframe}

\begin{contentframe}
    \frametitle{Architettura AI generativa}
    \framesubtitle{Embedding}
    
    \begin{itemize}
        \item I calcoli matematici non sono eseguiti direttamente sul testo, ma sul loro \textit{embedding}
    \end{itemize}

    \begin{block}{Embedding}
        Rappresentazione numerica del significato delle parole

        \begin{itemize}
            \item Ogni parola viene rappresentata come una serie di numeri
            \item Ogni numero indica la rilevanza di un certo tipo di concetto per quella parola
        \end{itemize}
    \end{block}
\end{contentframe}

\begin{exampleframe}
    \frametitle{Embedding (1)}

    \begin{table}
        \centering
        \begin{tabular}{|l|c|c|c|c|c|}
            \toprule
            Parola & Domestico & Animale & Taglia & Affettuoso & Caccia \\
            \midrule
            Cane        & 0.9 & 1.0 & 0.5 & 0.7 & 0.6 \\
            Gatto       & 0.9 & 1.0 & 0.3 & 0.8 & 0.4 \\
            Cavallo     & 0.5 & 1.0 & 0.9 & 0.5 & 0.3 \\
            Leone       & 0.1 & 1.0 & 0.8 & 0.2 & 1.0 \\
            Coniglio    & 0.8 & 1.0 & 0.2 & 0.8 & 0.3 \\
            Elefante    & 0.0 & 1.0 & 1.0 & 0.4 & 0.2 \\
            Tigre       & 0.0 & 1.0 & 0.8 & 0.3 & 0.8 \\
            Pappagallo  & 0.7 & 1.0 & 0.1 & 0.9 & 0.1 \\
            \bottomrule
        \end{tabular}
    \end{table}
\end{exampleframe}

\begin{exampleframe}
    \frametitle{Embedding (2)}

    \begin{table}
        \centering
        \begin{tabular}{|l|c|c|c|c|c|}
            \toprule
            Parola & Domestico & Animale & Taglia & Affettuoso & Caccia \\
            \midrule
            Cane        & 0.9 & 1.0 & 0.5 & 0.7 & 0.6 \\
            Gatto       & 0.9 & 1.0 & 0.3 & 0.8 & 0.4 \\
            Automobile  & 0.0 & 0.0 & 0.8 & 0.0 & 0.0 \\
            Fucile      & 0.0 & 0.0 & 0.6 & 0.0 & 0.9 \\
            Peluche     & 0.9 & 0.3 & 0.3 & 0.8 & 0.0 \\
            \bottomrule
        \end{tabular}
    \end{table}
\end{exampleframe}

\begin{contentframe}
    \frametitle{Architettura AI generativa}
    \framesubtitle{Tokenization}

    \begin{block}{Embedding}
        Rappresentazione numerica del significato delle parole
    \end{block}

    \begin{itemize}
        \item L'embedding richiede però che le frasi siano già state separate correttamente in una sequenza di parole
        \item Questa operazione si chiama \textit{Tokenization}
    \end{itemize}

    \begin{block}{Tokenization}
        Suddivisione di una frase in Token
        
        Un Token equivale in genere ad una parola, ma ci possono essere eccezioni
    \end{block}
\end{contentframe}

\begin{exampleframe}
    \frametitle{Tokenization}

    \image*[.95]{tokenization.png}

    \bigskip
    Fonte: \url{https://platform.openai.com/tokenizer}
\end{exampleframe}

\begin{contentframe}
    \frametitle{Architettura AI generativa}
    \framesubtitle{Riassumendo...}

    \smartartsequence[Prompt,Token +\\Embedding,Transformer,Risposta][2.2]

    \begin{itemize}
        \item \textbf{Prompt:} richiesta dell'utente
        \begin{itemize}
            \item La richiesta è composta da tutto lo storico dei messaggi della conversazione corrente
        \end{itemize}
        \item \textbf{Tokenization + embedding:} separazione del testo in token ($\sim$ parole) e conversione in numeri
        \begin{itemize}
            \item Ogni numero rapprensenta la rilevanza della parola per un certo concetto
        \end{itemize}
        \item \textbf{Transformer:} calcoli matematici per generare la risposta
    \end{itemize}
\end{contentframe}

\begin{exampleframe}
    \frametitle{Approfondimento}

    \begin{itemize}
        \item \href{https://github.com/DavidePonzini/didattica/raw/refs/heads/main/res/HowWordVectorsEncodeMeaning.mkv}{How word vectors encode meaning \faFileVideo}
    \end{itemize}
\end{exampleframe}

\begin{contentframe}
    \frametitle{Utilizzi principali}

    \begin{itemize}
        \item Lo strumento è nato per un numero ristretto di operazioni testuali:
        \begin{itemize}
            \item Text Summarization
            \item Information Extraction
            \item Question Answering
            \item Text Classification
        \end{itemize}
        
        \bigskip
        \item Successivamente, si è notato che tramite tecniche di \textbf{prompting}, lo strumento è in grado di risolvere un ampio numero di compiti
    \end{itemize}
\end{contentframe}

\makesectionframe{Limitazioni e rischi}

\begin{contentframe}
    \frametitle{Limitazioni AI generativa (1)}

    \begin{itemize}
        \item La sua conoscenza deriva solo dai dati su cui è stato fatto training
        \begin{itemize}
            \item Non ha le informazioni più recenti
        \end{itemize}
        
        \bigskip
        \item Può estrarre e rielaborare informazioni da una frase, ma non la capisce davvero
        \begin{itemize}
            \item È uno strumento, non una persona
        \end{itemize}
        
        \bigskip
        \item Gli strumenti sono pensati per dire ciò che l'utente si vuol \textit{(inconsciamente)} sentire dire
        \begin{itemize}
            \item Molto spesso corrisponde al vero, ma non è sempre detto!
            \item Esempio: Paolo Rossi e suo ``padre'' \href{https://chatgpt.com/share/6788f77b-4758-8003-84ec-12aefa6654a0}{(1)}, \href{https://chatgpt.com/share/6788f901-f5a4-8003-8001-8f26228454e3}{(2)}, \href{https://chatgpt.com/share/679892a1-cae4-8003-a0f0-aa8c36fd2d0d}{(3)}
        \end{itemize}
    \end{itemize}
\end{contentframe}

\begin{contentframe}
    \frametitle{Limitazioni AI generativa (2)}

    \begin{itemize}
        \item Le risposte sono generate in maniera probabilistica
        \begin{itemize}
            \item È probabile ottenere la risposta corretta, ma non è \underline{mai} garantito!
            \item La stessa domanda può fornire risposte diverse, a volte anche significativamente
        \end{itemize}

        \bigskip
        \item Esempio: calcolo del dominio di $\frac{\sqrt{-5x + 7}}{\ln(\sqrt{x})} + \cos^2\left(\frac{\pi}{4x}\right)$
        \begin{itemize}
            \item \href{https://chatgpt.com/share/6776910a-2990-8003-a158-e9337a55edf2}{Risposta 1}
            \item \href{https://chatgpt.com/share/67769686-f368-8003-b862-005258d4ba3a}{Risposta 2}
            \item \href{https://www.wolframalpha.com/input?i2d=true&i=Divide\%5BSqrt\%5B-5x\%2B7\%5D\%2Cln\%5C\%2840\%29Sqrt\%5Bx\%5D\%5C\%2841\%29\%5D\%2BSquare\%5Bcos\%5C\%2840\%29Divide\%5B\%CF\%80\%2C4x\%5D\%5C\%2841\%29\%5Ddomain}{Soluzione Wolfram Alpha}
        \end{itemize}

    \end{itemize}
\end{contentframe}


\begin{contentframe}
    \frametitle{Bias}

    \begin{itemize}
        \item Problema particolarmente rilevante nel Machine Learning.

        \bigskip
        \item Si verifica quando sono presenti dati ``di parte'' nella fase di training del modello

        \bigskip
        \item Questi pregiudizi riaffiorano nelle risposte
        \item Spesso sono difficili da notare
    \end{itemize}
\end{contentframe}

\begin{exampleframe}
    \frametitle{Bias}
    \framesubtitle{Esempi}

    \centering
    \begin{itemize}
        \item Il modello è allenato (anche) su dati di utenti americani
    \end{itemize}
    
    \image*[.6]{bias_1.jpg}
\end{exampleframe}

\begin{exampleframe}
    \frametitle{Bias}
    \framesubtitle{Esempi}

    \centering
    \begin{itemize}
        \item I dati sui cui è allenato il modello contengono diversi tipi di pregiudizi

        \bigskip
        \item \href{https://chatgpt.com/share/67518d78-2588-8003-94c1-0c0a474b0fe7}{Scrivere una mail}
    \end{itemize}
\end{exampleframe}

\begin{contentframe}
    \frametitle{Data Leak \& Data Breach}

    \begin{itemize}
        \item \textbf{Data Leak}: esposizione a dati sensibili da parte di un sistema AI
        \begin{itemize}
            \item Inizia quando si scrivono informazioni sensibili nelle richieste all'AI
            \item Queste informazioni vengono integrate nel modello e possono comparire in futuro nelle risposte che fornisce a tutti
        \end{itemize}

        \bigskip
        \item \textbf{Data Breach}: utilizzo dei dati ottenuti tramite Data Leak per azioni dannose
        \begin{itemize}
            \item Furti d’identità, ricatti, danni economici o reputazionali, diffusione di fake news, ...
        \end{itemize}
    \end{itemize}
\end{contentframe}

\begin{exampleframe}
    \frametitle{Data Leak \& Data Breach}
    \framesubtitle{Esempi}

    \begin{itemize}
        \item \href{https://www.reddit.com/r/ProgrammerHumor/comments/17co7vv/hackingin2024/}{Password root Server Google}
        \item \href{https://www.reddit.com/r/ChatGPT/comments/14bpla2/thanks_grandma_one_of_the_keys_worked_for_windows/}{Chiavi attivazione Windows 7 Ultimate}
    \end{itemize}
\end{exampleframe}

