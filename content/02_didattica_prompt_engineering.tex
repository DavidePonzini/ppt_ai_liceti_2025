\makesectionframe{AI generativa nella didattica}[Rischi e opportunità]

\begin{contentframe}
    \frametitle{AI generativa nella didattica}

    \begin{itemize}
        \item La diffusione dell'AI generativa sta avendo un forte impatto in ambito didattico

        \bigskip
        \item Le conseguenze non sono ancora chiarissime
        \item Esistono sia rischi che opportunità
        \item Dipendono da come si usa lo strumento
    \end{itemize}
\end{contentframe}

\begin{contentframe}
    \frametitle{AI generativa nella didattica}

    \begin{itemize}
        \item Rischio più comune: affidarsi eccessivamente allo strumento
        \item Bisogna sempre ricordare che le risposte generate potrebbero:
        \begin{itemize}
            \item Essere evidentemente errate \textit{(molto raro, ma possibile)}
            \item Sembrare plausibili, ma contenere informazioni errate \textit{(poco frequente, ma avviene)}
            \item Essere prevalentemente corrette, ma con qualche errore o imprecisione difficile da notare \textit{(più frequente di quello che si pensa)}
            \item Essere corrette ma incomplete \textit{(mediamente frequente)}
            \item Essere corrette e complete \textit{(mediamente frequente)}
        \end{itemize}

        \bigskip
        \item Soluzione: essere consapevoli dei limiti dello strumento
    \end{itemize}
\end{contentframe}

\begin{exerciseframe}
    \frametitle{AI generativa per studenti}
    \framesubtitle{Esempio interattivo}

    \begin{enumerate}
        \item Svolgere il seguente compito \underline{usando} AI generativa
        \begin{itemize}
            \item Tempo a disposizione: 6 min
            \item Per ogni domanda rispondere scrivendo circa 10 righe
            \item \url{https://forms.gle/k3E9nKsYXwLWeowR7}
        \end{itemize}

        \pause
        \bigskip
        \item Rispondere al \href{https://play.blooket.com/host?id=678cef9809b01a90328de16a}{seguente quiz}
    \end{enumerate}
\end{exerciseframe}

\begin{contentframe}
    \frametitle{Rischi per studenti}

    \begin{columns}
        \col{.5}
        \begin{itemize}
            \item Riduzione apprendimento
            \item Distorsione apprendimento
            \item Riduzione sviluppo pensiero critico
            \item Dipendenza dallo strumento
        \end{itemize}
        
        \col{.5}
        \centering
        \image{student_sleeping.jpg}
    \end{columns}
\end{contentframe}

\begin{contentframe}
    \frametitle{Opportunità per studenti}

    \begin{columns}
        \col{.5}
        \begin{itemize}
            \item Supporto all'apprendimento
            \item Sviluppo pensiero critico
            \item Accesso a contenuti personalizzati
            \item Accesso a informazioni non reperibili su internet
            \item Aumento motivazione
            \item Disponibilità 24/7
        \end{itemize}
        
        \col{.5}
        \centering
        \image{student_smart.jpg}
    \end{columns}
\end{contentframe}

\begin{contentframe}
    \frametitle{Rischi per studenti}
    \framesubtitle{Riduzione apprendimento}

    \begin{itemize}
        \item Gli studenti potrebbero limitarsi a utilizzare l'output dell'AI senza comprendere a fondo il materiale

        \bigskip
        \item Non ``faticando'' per risolvere i problemi si impara di meno (o per niente)

        \bigskip
        \item \textit{Esempio: farsi fare i compiti e copiare meccanicamente le risposte}
    \end{itemize}
\end{contentframe}

\begin{contentframe}
    \frametitle{Rischi per studenti}
    \framesubtitle{Distorsione apprendimento}

    \begin{itemize}
        \item I bias e gli errori presenti nelle risposte potrebbero essere ``imparati'' dagli studenti

        \bigskip
        \item Le risposte errate potrebbero essere interpretate come corrette

        \bigskip
        \item \textit{Esempio: usare AI come dei motori di ricerca}
    \end{itemize}
\end{contentframe}

\begin{contentframe}
    \frametitle{Rischi per studenti}
    \framesubtitle{Riduzione sviluppo pensiero critico}

    \begin{itemize}
        \item Avere immediatamente una risposta a qualsiasi domanda porta a smettere di ragionare
        \item Particolarmente rilevante in soggetti più giovani (specialmente elementari/medie)

        \bigskip
        \item Senza pensiero critico non siamo in grado di distinguere le risposte corrette da quelle errate o imprecise

        \bigskip
        \item \textit{Esempio: assumere che tutte le risposte fornite dallo strumento siano corrette o che lo strumento sappia meglio di noi}
    \end{itemize}
\end{contentframe}

\begin{contentframe}
    \frametitle{Rischi per studenti}
    \framesubtitle{Dipendenza dallo strumento}

    \begin{itemize}
        \item Affidandosi sempre all'AI per ogni singolo compito, si diventa incapaci di svolgere attività senza

        \bigskip
        \item Non si sviluppano molte abilità utili per la vita
        
        \bigskip
        \item \textit{Esempio: affidarsi ad ogni minima difficoltà al supporto dell'AI}
    \end{itemize}
\end{contentframe}

\begin{contentframe}
    \frametitle{Opportunità per studenti}
    \framesubtitle{Supporto all'apprendimento}

    \begin{itemize}
        \item Un utilizzo consapevole può portare ad apprendere meglio concetti noti
        \item Un utilizzo consapevole può portare a scoprire nuovi concetti

        \bigskip
        \item \textit{Esempio: chiedere chiarimenti e spiegazioni o effettuare ricerche}
        \item Ricordarsi di verificare sempre le risposte ottenute!
    \end{itemize}
\end{contentframe}

\begin{contentframe}
    \frametitle{Opportunità per studenti}
    \framesubtitle{Sviluppo pensiero critico}

    \begin{itemize}
        \item Tutte le risposte fornite da AI andrebbero messe in dubbio, in quanto non sono verificate
        \item Questo processo porta allo sviluppo del pensiero critico

        \bigskip
        \item \textit{Esempio: ``In equitazione, il trotto è l'andatura più usata nel salto degli ostacoli''}
    \end{itemize}
\end{contentframe}

\begin{contentframe}
    \frametitle{Opportunità per studenti}
    \framesubtitle{Accesso a contenuti personalizzati}

    \begin{itemize}
        \item I contenuti creati da AI possono essere facilmente adattati alle necessità o agli interessi degli studenti
        \item Le lezioni possono diventare più coinvolgenti, usando argomenti verso cui gli studenti hanno interesse personale

        \bigskip
        \item \textit{Esempio: un problema di matematica può essere personalizzato in base agli sport frequentati dai singoli studenti}
    \end{itemize}
\end{contentframe}

\begin{exerciseframe}
    \frametitle{Accesso a contenuti personalizzati}
    \framesubtitle{Esercizio}

    \begin{itemize}
        \item Farsi spiegare un argomento con esempi pratici inerenti ad un vostro interesse
        \begin{itemize}
            \item \textit{``Spiega le frazioni con esempi pratici relativi alla cucina''}
            \item \textit{``Spiega la probabilità condizionata includendo esempi relativi ai videogiochi''}
            \item \textit{``Spiega le principali differenze tra la cultura occidentale e quella asiatica. Fornisci esempi relativi alle arti marziali''}
        \end{itemize}
    \end{itemize}
\end{exerciseframe}

\begin{contentframe}
    \frametitle{Opportunità per studenti}
    \framesubtitle{Accesso a informazioni non reperibili su internet}

    \begin{itemize}
        \item Nel caso su internet non si trovino risposte ad una certa domanda, lo strumento potrebbe generare comunque una risposta utile

        \bigskip
        \item Utile specialmente per scenari molto specifici o di nicchia
    \end{itemize}
\end{contentframe}

\begin{contentframe}
    \frametitle{Opportunità per studenti}
    \framesubtitle{Aumento motivazione}

    \begin{itemize}
        \item Utilizzare strumenti digitali può essere considerata attività ludica/ricreativa
        \begin{itemize}
            \item Se fatta nei modi e tempi appropriati, può aumentare la motivazione ad apprendere negli studenti
        \end{itemize}
        
        \bigskip
        \item Fare attenzione a circorscrivere bene l'attività ed ad inquadrarla in un contesto più grande

        \bigskip
        \item \textit{Esempio: pensare all'utilizzo di Kahoot. Si dedica un momento preciso della lezione ad usare AI generativa, con l'obiettivo di apprendere meglio i concetti trattati}
    \end{itemize}
\end{contentframe}

\begin{contentframe}
    \frametitle{Opportunità per studenti}
    \framesubtitle{Disponibilità 24/7}

    \begin{itemize}
        \item Lo strumento è disponibile in qualsiasi momento
        \bigskip
        \item I tempi di attesa sono ridotti a pochi secondi
    \end{itemize}
\end{contentframe}

\begin{contentframe}
    \frametitle{AI generativa per docenti}

    \begin{columns}
        \col{.5}
        \begin{itemize}
            \item AI generativa non è limitata agli studenti
            \item Anche i docenti possono usare questi strumenti nelle loro attività
    
            \bigskip
            \item Si può usare in tutti i modi descritti per gli studenti
            \begin{itemize}
                \item Sono presenti gli stessi rischi!
            \end{itemize}
            \item Si può usare anche a supporto delle lezioni
        \end{itemize}
        
        \col{.5}
        \centering
        \image{teacher_student.jpg}
    \end{columns}
\end{contentframe}

% \begin{exerciseframe}
%     \frametitle{AI generativa per docenti}
%     \framesubtitle{Esempio interattivo}

%     Formulando esclusivamente prompt semplici:
%     \begin{enumerate}
%         \item Far generare esercizi per una verifica
%         \begin{itemize}
%             \item \textit{``Prepara degli esercizi di ripasso per una verifica sui Promessi Sposi''}
%         \end{itemize}

%         \pause
%         \bigskip
%         \item Correggere esercizi svolti da studenti ed assegnare un voto
%         \begin{itemize}
%             \item \textit{``Correggi il seguente compito: ...''}
%             \item \textit{``Fornisci un voto in decimi per il seguente compito: ...''}
%         \end{itemize}
%     \end{enumerate}
% \end{exerciseframe}

\begin{contentframe}
    \frametitle{Rischi per docenti}

    \begin{columns}
        \col{.5}
        \begin{itemize}
            \item Generazione di materiali o esercizi banali/superficiali
            \item Giudizi non coerenti o pertinenti
        \end{itemize}
    
        \col{.5}
        \centering
        \image{teacher_bad.jpg}
    \end{columns}
\end{contentframe}

\begin{contentframe}
    \frametitle{Opportunità per docenti}

    \begin{columns}
        \col{.5}
        \begin{itemize}
            \item Risparmio di tempo \textit{(soprattutto per operazioni meccaniche)}
            \item Analisi più approfondite rispetto a strumenti tradizionali
            \item Generazione materiale personalizzato
            \item Generazione feedback personalizzato
            \item Revisione materiale creato
        \end{itemize}
        
        \col{.5}
        \centering
        \image{teacher_good.jpg}
    \end{columns}
\end{contentframe}

\begin{contentframe}
    \frametitle{Rischi per docenti}
    \framesubtitle{Generazione di materiali o esercizi banali/superficiali}

    \begin{itemize}
        \item Gli esercizi generati tramite richieste semplici tendono:
        \begin{itemize}
            \item Ad avere difficoltà medio-bassa
            \item A non essere coerenti col programma
            \item A non ricoprire tutti gli aspetti rilevanti
        \end{itemize}

        \bigskip
        \item Con richieste più elaborate si possono ottenere buoni risultati
        
        \bigskip
        \item \textit{Esempio: ``genera degli esercizi sui verbi in inglese''}
    \end{itemize}
\end{contentframe}

\begin{contentframe}
    \frametitle{Rischi per docenti}
    \framesubtitle{Giudizi non coerenti o pertinenti}

    \begin{itemize}
        \item Correzioni e valutazioni potrebbero:
        \begin{itemize}
            \item Non essere esaustive
            \item Non essere coerenti col programma
            \item Non essere coerenti tra di loro
        \end{itemize}

        \bigskip
        \item Richieste elaborate mitigano questo problema ma non lo eliminano
    \end{itemize}
\end{contentframe}

\begin{contentframe}
    \frametitle{Opportunità per docenti}
    \framesubtitle{Risparmio di tempo}

    \begin{itemize}
        \item È possibile generare molto velocemente contenuti banali

        \bigskip
        \item Molto utile per la burocrazia
        \item Utile per generare una traccia da adattare successivamente
    \end{itemize}
\end{contentframe}

\begin{contentframe}
    \frametitle{Opportunità per docenti}
    \framesubtitle{Analisi più approfondite rispetto a strumenti tradizionali}

    \begin{itemize}
        \item È possibile analizzare molto velocemente anche aspetti complessi di un elaborato

        \bigskip
        \item Attenzione: soprattutto con prompt semplici, non è garantito che le analisi siano accurate o comprensive

        \bigskip
        \item \textit{Esempio: ``descrivi la padronanza del linguaggio tecnico in questo elaborato''}
    \end{itemize}
\end{contentframe}

\begin{contentframe}
    \frametitle{Opportunità per docenti}
    \framesubtitle{Generazione materiale personalizzato}

    \begin{itemize}
        \item Si possono generare velocemente materiali personalizzati in base alle esigenze
        \item Come sempre, è necessario controllare manualmente quanto prodotto

        \bigskip
        \item \textit{Esempio: ``genera uno schema riassuntivo dei principali tipi di proposizione''} 
    \end{itemize}
\end{contentframe}

\begin{contentframe}
    \frametitle{Opportunità per docenti}
    \framesubtitle{Generazione feedback personalizzato}

    \begin{itemize}
        \item Si possono generare velocemente feedback personalizzati in base agli elaborati prodotti ed alle esigenze del docente

        \bigskip
        \item \textit{Esempio: \href{https://chatgpt.com/share/6776a77f-8288-8003-86ad-95b7b556fcf6}{Feedback elaborato}}
    \end{itemize}
\end{contentframe}

\begin{contentframe}
    \frametitle{Opportunità per docenti}
    \framesubtitle{Revisione materiale creato}

    \begin{itemize}
        \item Si può usare l'AI per migliorare il materiale creato:
        \begin{itemize}
            \item Chiedendo un parere/giudizio
            \item Chiedendo di suggerire miglioramenti
        \end{itemize}

        \bigskip
        \item Si può chiedere di risolvere gli esercizi e vedere se ha capito correttamente la consegna
    \end{itemize}
\end{contentframe}

% \begin{exerciseframe}
%     \frametitle{AI generativa per docenti}
%     \framesubtitle{Esempio interattivo}

%     Formulando in maniera elaborata i prompt:
%     \begin{enumerate}
%         \item Far generare esercizi in preparazione ad una verifica
%         \begin{itemize}
%             \item Esempio: \href{https://chatgpt.com/share/678ba114-e39c-8003-9f3e-cba11d1fea00}{Promessi Sposi}
%         \end{itemize}

%         \bigskip
%         \item Far generare esercizi da includere all'interno di una verifica
        
%         \bigskip
%         \item Correggere esercizi svolti da studenti ed assegnare un voto
%         \begin{itemize}
%             \item \textit{``Correggi il seguente compito, usando i seguenti criteri ed applicando la seguente griglia di valutazione: ...''}
%         \end{itemize}
%     \end{enumerate}
% \end{exerciseframe}

\begin{contentframe}
    \frametitle{Esplorazione}

    \begin{itemize}
        \item Tramite AI generativa è possibile esplorare ed imparare concetti sconosciuti
        \item \underline{È fondamentale essere guidati dalla curiosità}
        \begin{itemize}
            \item La prima risposta non sarà mai esaustiva, bisogna continuare a chiedere domande
            \item L'obiettivo non deve essere fare un certo numero di domande, ma tentare di capire quanto più possibile
        \end{itemize}

        \bigskip
        \item \textit{Vediamo un esempio...}
    \end{itemize}
    
\end{contentframe}

\begin{exerciseframe}
    \frametitle{Esplorazione}
    \framesubtitle{Esercizio}

    \begin{itemize}
        \item Scoprire un argomento nuovo con cui non siete familiari
        \begin{itemize}
            \item Tempo a disposizione: almeno 10 min
            \item Argomento:\pause
                \begin{CJK}{UTF8}{min}
                    Kana (ひらがな・カタカナ)
                \end{CJK}
        \end{itemize}

        \pause
        \bigskip
        \item Svolgere un \href{https://play.blooket.com/host?id=67855de45130aefd0c2e869f}{quiz} sui concetti appena appresi
    \end{itemize}
\end{exerciseframe}

\makesectionframe{Prompt engineering}

\begin{contentframe}
    \frametitle{I prompt}

    \begin{itemize}
        \item \textbf{Prompt:} richieste che faccio ad una AI generativa
        \item \textbf{Prompt Engineering:} tecniche per formulare le richieste in modo ottimale

        \bigskip
        \item Tramite tecniche di prompt engineering, posso usare lo strumento per ottenere buoni risultati in tanti campi 
    \end{itemize}
\end{contentframe}

\begin{contentframe}
    \frametitle{Struttura dei prompt}

    \begin{columns}
        \col{.5}
        \begin{itemize}
            \item Lo strumento genera il messaggio a partire dai precedenti
    
            \bigskip
            \item Tre tipi di messaggio
            \begin{itemize}
                \item \textbf{Sistema:} istruzioni per il modello \textit{(utile solo in situazioni particolari)}
                \item \textbf{Utente:} richiesta dell'utente
                \item \textbf{Assistente:} risposta del modello
            \end{itemize}
        \end{itemize}

        \col{.5}
        \centering
        \smartartflowvertical[System,User,Assistant]
    \end{columns}
\end{contentframe}

\begin{exampleframe}
    \frametitle{Struttura dei prompt}
    \framesubtitle{Esempio}

    \begin{itemize}
        \item \href{https://chatgpt.com/share/6759a21f-19ec-8003-a172-e0630eae761d}{Prompt 1} ( $\boxed{S + U} \rightarrow A$ )
        \item \href{https://chatgpt.com/share/67605a8c-2fec-8003-8e82-c54f49687c1b}{Prompt 2} ( $\boxed{S + U_1 + A_1 + U_2} \rightarrow A_2$ )

        \bigskip
        \item Il prompt di sistema non è specificato, quindi è:\\
            \texttt{You are ChatGPT, a large language model trained by OpenAI. Answer as concisely and informatively as possible. Your knowledge is up to date until [cutoff date].}
    \end{itemize}
\end{exampleframe}

\makesectionframe{Come si scrivono i prompt}

\begin{contentframe}
    \frametitle{Come si scrivono i prompt}

    \begin{itemize}
        \item \textbf{Obiettivo:} aumentare la possibilità di ottenere \underline{esattamente} la risposta desiderata
        \begin{enumerate}%[label=\alph*)]
            \item Specificità e chiarezza
            \item Formati input e output
            \item Utilizzo di delimitatori
            \item Decomposizione attività complesse
        \end{enumerate}
        
        \bigskip
        \item \textbf{Assicurarsi di poter sempre verificare le risposte fornite!}
        \begin{itemize}
            \item Con conoscenza personale
            \item Tramite strumenti esterni
        \end{itemize}
    \end{itemize}
\end{contentframe}

\begin{contentframe}
    \frametitle{Come si scrivono i prompt}
    \framesubtitle{Specificità e chiarezza}

    \begin{itemize}
        \item Descrivi chiaramente l'obiettivo desiderato

        \bigskip
        \item Evita ambiguità per evitare risposte non pertinenti
    \end{itemize}
\end{contentframe}

\begin{exampleframe}
    \frametitle{Come si scrivono i prompt}
    \framesubtitle{Specificità e chiarezza}

    \begin{itemize}
        \item \href{https://chatgpt.com/share/6759a98f-8218-8003-9eb8-bf85e9699044}{Istruzioni vaghe -- studenti}
        \item \href{https://chatgpt.com/share/6759aa0f-1404-8003-b15e-0330328e1803}{Richiesta precisa e ben definita -- studenti}

        \bigskip
        \item \href{https://chatgpt.com/share/6759aa68-0ba4-8003-ae44-d856b0014d10}{Istruzioni vaghe -- docenti}
        \item \href{https://chatgpt.com/share/6759ab31-c6c0-8003-a3aa-8c06ddf5ee13}{Richiesta precisa e ben definita -- docenti}
    \end{itemize}
\end{exampleframe}

\begin{contentframe}
    \frametitle{Come si scrivono i prompt}
    \framesubtitle{Formati input e output}

    \begin{itemize}
        \item Usare formati strutturati per dati di input migliora la comprensione dei dati

        \bigskip
        \item Specifica in che formato vuoi ottenere l'output \textit{(paragrafo, lista, tabella, ...)}
    \end{itemize}
\end{contentframe}

\begin{exampleframe}
    \frametitle{Formati input e output}
    \framesubtitle{Formato non strutturato}

    \justifying
    \textit{Per una carbonara per due persone ti serve un bel pezzo di guanciale, diciamo intorno ai 100 grammi, tagliato a listarelle. Poi ci vogliono le uova: un tuorlo a testa più un uovo intero in tutto va benissimo. Non dimenticare il pecorino romano, almeno 50 grammi a testa, grattugiato fresco. Abbondante pepe nero macinato al momento, ovviamente, e la pasta: 100 grammi di spaghetti o rigatoni per ogni persona. Tutto qui, niente panna, niente aglio, solo il necessario!}
\end{exampleframe}

\begin{exampleframe}
    \frametitle{Formati input e output}
    \framesubtitle{Formato strutturato}

    \begin{table}[h!]
        \centering
        \begin{tabular}{lc}
            \toprule
            \textbf{Ingrediente}            & \textbf{Quantità per 2 persone}   \\ 
            \midrule
            Guanciale                       & 100 g                             \\ 
            Uova                            & 2 tuorli + 1 uovo                 \\ 
            Pecorino romano                 & 100 g                             \\ 
            Pepe nero                       & q.b.                              \\ 
            Pasta (spaghetti o rigatoni)    & 200 g                             \\ 
            \bottomrule
        \end{tabular}
    \end{table}
\end{exampleframe}

        


\begin{exampleframe}
    \frametitle{Come si scrivono i prompt}
    \framesubtitle{Formati input e output}

    \begin{itemize}
        \item \href{https://chatgpt.com/share/6759ada4-a0c8-8003-9a64-fcbce8b7c494}{Elenco}
        \item \href{https://chatgpt.com/share/67655f30-2080-8003-85d4-f5b3b3540716}{Tabella}
        \item \href{https://chatgpt.com/share/67655f61-6fb0-8003-9cb4-5fc6b04dcee5}{CSV}
        \item \href{https://chatgpt.com/share/67655f87-b9f0-8003-8266-7729917000d8}{JSON}
    \end{itemize}
\end{exampleframe}

\begin{contentframe}
    \frametitle{Come si scrivono i prompt}
    \framesubtitle{Utilizzo dei delimitatori}

    \begin{itemize}
        \item Separa chiaramente i dati della richiesta (se possibile)

        \bigskip
        \item Usa caratteri speciali come delimitatori dei vari elementi della richiesta
        \begin{itemize}
            \item \texttt{--- Richiesta ---}\\
                \textit{Calcola il prezzo delle mele}\\
                \texttt{--- Dati ---}\\
                \textit{Mele acquistate: 13}\\
                \textit{(altri dati)}

            \bigskip
            \item \texttt{Q:} \textit{Calcola il prezzo delle mele sapendo che ...}\\
                \texttt{A:}
        \end{itemize}
    \end{itemize}
\end{contentframe}

\begin{exampleframe}
    \frametitle{Come si scrivono i prompt}
    \framesubtitle{Utilizzo dei delimitatori}

    \begin{itemize}
        \item \href{https://chatgpt.com/share/675af550-6bb0-8003-966d-b7a36cb9fb6e}{Estrarre dati dai problemi}
        \begin{itemize}
            \item Non è sempre necessario, ma aiuta lo strumento a non confondersi
        \end{itemize}
    \end{itemize}
\end{exampleframe}

\begin{contentframe}
    \frametitle{Come si scrivono i prompt}
    \framesubtitle{Decomposizione attività complesse}

    \begin{itemize}
        \item Suddividi compiti complessi in sottoproblemi

        \bigskip
        \item Invece di richiedere la risoluzione di tanti problemi contemporaneamente, chiedi un problema alla volta
    \end{itemize}
\end{contentframe}

\begin{exampleframe}
    \frametitle{Come si scrivono i prompt}
    \framesubtitle{Decomposizione attività complesse}

    \begin{itemize}
        \item \href{https://chatgpt.com/share/675be9d5-8804-8003-b24a-851e8edf3401}{Dieci problemi in un prompt}
        \item \href{https://chatgpt.com/share/675bef6b-fa34-8003-8932-b1b662061efe}{Un solo problema per prompt}

        \bigskip
        \item Questo principio vale anche se abbiamo un singolo problema composto da tanti passi
    \end{itemize}
\end{exampleframe}

\begin{exerciseframe}
    \frametitle{Esercizio}

    Provare a scrivere dei prompt per:
    \begin{enumerate}
        \item Spiegare un certo concetto
        \item Scrivere una mail su un certo argomento e con un certo tono
        \item Rispondere alle seguenti \href{https://raw.githubusercontent.com/DavidePonzini/didattica/refs/heads/main/quesiti_medie.md}{domande}
    \end{enumerate}

    \bigskip
    Sperimentare con diversi tipi di prompt e valutare le differenze nelle risposte
\end{exerciseframe}

\makesectionframe{Quali prompt usare?}

\begin{contentframe}
    \frametitle{Quali prompt usare?}

    \begin{columns}
        \col{.5}
        \begin{itemize}
            \item Regole importanti prima di iniziare:
            \begin{enumerate}
                \item Non avere paura di chiedere chiarimenti 
                \item Non avere paura di dialogare con lo strumento
                \item Non avere paura di faticare
            \end{enumerate}
        \end{itemize}
        
        \col{.5}
        \centering
        \image[1][.6]{talk_ai.jpg}
    \end{columns}
\end{contentframe}

\begin{contentframe}
    \frametitle{Quali prompt usare?}
    \framesubtitle{Concetti sconosciuti}
    
    \begin{itemize}
        \item \textbf{Se nella risposta sono presenti elementi che non conosciamo...}

        \bigskip
        \item Chiedere spiegazioni
        \begin{itemize}
            \item Utile se abbiamo compreso quasi tutta la risposta ma non conosciamo alcuni concetti
        \end{itemize}
        
        \item Chiedere di riscrivere in maniera più semplice
        \begin{itemize}
            \item Utile se abbiamo capito poco o nulla
        \end{itemize}
    \end{itemize}
\end{contentframe}

\begin{exampleframe}
    \frametitle{Quali prompt usare?}
    \framesubtitle{Concetti sconosciuti}

    \begin{itemize}
        \item Chiedere spiegazioni
        \begin{itemize}
            \item \textit{Che cosa è la rapallizzazione?}
            \item \textit{A cosa serve il wok?}
            \item \textit{Cosa fa il BIOS?}
        \end{itemize}
    \end{itemize}
\end{exampleframe}

\begin{exampleframe}
    \frametitle{Quali prompt usare?}
    \framesubtitle{Concetti sconosciuti}

    \begin{itemize}
        \item Chiedere di riscrivere in maniera più semplice
        \begin{itemize}
            \item \textit{Semplifica la risposta precedente}
            \item \textit{Riscrivi come se dovessi spiegarlo ad un giovane studente}
            \item \textit{Riscrivi come se dovessi spiegarlo ad un ragazzo di 15 anni\footnote[frame]{ChatGPT non conosce la vostra età, quindi potrebbe fornire una risposta con un\\tono adatto per un adulto o un esperto del settore}}
            \item \textit{Riscrivi come se dovessi spiegarlo ad un bambino di [8 / 5 / 3] anni}
        \end{itemize}
    \end{itemize}
\end{exampleframe}

\begin{exampleframe}
    \frametitle{Esempi}

    \begin{itemize}
        \item \href{https://chatgpt.com/share/675c1a86-56c4-8003-bd9b-c737657e714a}{Analisi linguistica testo}
        % \item \href{https://chatgpt.com/share/675c1a2c-380c-8003-a32f-8eac1a680302}{Python (list comprehension)}
    \end{itemize}
\end{exampleframe}

\begin{contentframe}
    \frametitle{Quali prompt usare?}
    \framesubtitle{Elementi diversi dalle aspettative}
    
    \begin{itemize}
        \item \textbf{Se nella risposta sono presenti elementi che non ci aspettavamo...}

        \bigskip
        \item Chiedere informazioni sulla scelta
        \begin{itemize}
            \item Utile se stiamo cercando una soluzione qualsiasi al problema
            \item Attenzione a formulare la domanda in maniera neutra!
        \end{itemize}
        
        \item Chiedere alternative
        \begin{itemize}
            \item Utile se stiamo cercando una soluzione specifica al problema
        \end{itemize}
    \end{itemize}
\end{contentframe}

\begin{exampleframe}
    \frametitle{Quali prompt usare?}
    \framesubtitle{Elementi diversi dalle aspettative}

    \begin{itemize}
        \item Chiedere informazioni sulla scelta
        \begin{itemize}
            \item \textit{Perché hai usato la pancetta?}
            \item \textit{Perché hai usato la pancetta invece del guanciale?}
            \item \textcolor{red}{\st{\textit{Perché la pancetta è meglio del guanciale?}}}
            \begin{itemize}
                \item \textcolor{red}{Domanda di parte, fornisce risposte di parte}
            \end{itemize}
            \item \textit{Spiega le differenze tra usare la pancetta ed il guanciale per preparare la carbonara}
        \end{itemize}
    \end{itemize}
\end{exampleframe}

\begin{exampleframe}
    \frametitle{Quali prompt usare?}
    \framesubtitle{Elementi diversi dalle aspettative}

    \begin{itemize}
        \item Chiedere alternative
        \begin{itemize}
            \item \textit{Usa il guanciale invece della pancetta}
            \item \textit{Non usare la pancetta}
            \item \textit{Risolvi il problema in modo diverso, se possibile}
            \item \textit{Fornisci 3 modi diversi di risolvere il problema, se esistono}
        \end{itemize}
    \end{itemize}
\end{exampleframe}

\begin{contentframe}
    \frametitle{Quali prompt usare?}
    \framesubtitle{Mancata comprensione della richiesta}
    
    \begin{itemize}
        \item \textbf{Se sembra che lo strumento non abbia compreso appieno le nostre intenzioni...}

        \bigskip
        \item Provare a formulare la richiesta con altre parole
        \item Chiedere di concentrarsi su un certo aspetto della risposta
        \item Aggiungere informazioni utili per la risposta e chiedere di provare nuovamente
        \begin{itemize}
            \item Particolarmente utile per richieste su argomenti non banali o facilmente fraintendibili
        \end{itemize}
        \item Modificare la domanda precedente, aggiungendo informazioni rilevanti
    \end{itemize}
\end{contentframe}

\begin{contentframe}
    \frametitle{Quali prompt usare?}
    \framesubtitle{Difficoltà interpretazione problema da parte dell'utente}
    
    \begin{itemize}
        \item \textbf{Se non ci è chiaro come risolvere il problema...}

        \bigskip
        \item Chiedere di spiegarci i passi necessari alla risoluzione
        \begin{itemize}
            \item \textit{Non chiedere di risolverlo, ma di dire cosa fare per risolverlo}
        \end{itemize}
    \end{itemize}
\end{contentframe}

\begin{exampleframe}
    \frametitle{Quali prompt usare?}
    \framesubtitle{Difficoltà interpretazione problema da parte dell'utente}

    \begin{itemize}
        \item Chiedere di spiegarci i passi necessari alla risoluzione
        \begin{itemize}
            \item \textit{Quali passi sono necessari per risolvere un'equazione fratta?}
            \item \textit{Elenca le azioni necessarie per preparare le lasagne?}
            \item \textit{Da quali operazioni è composta la creazione di una campagna pubblicitaria di successo?}
            \item \textit{Da quali passaggi è formata la stesura di un articolo di giornale?}
        \end{itemize}
    \end{itemize}
\end{exampleframe}

\begin{contentframe}
    \frametitle{Quali prompt usare?}
    \framesubtitle{Risposta troppo generica}
    
    \begin{itemize}
        \item \textbf{Se la risposta è troppo generica o astratta...}

        \bigskip
        \item Fare una domanda più precisa
        \item Chiedere di risolvere solo una parte del problema \textit{(tipo decomposizione vista prima)}
        \item Fornire informazioni rilevanti sul materiale a nostra disposizione
        \item Chiedere esempi
    \end{itemize}
\end{contentframe}

\begin{exampleframe}
    \frametitle{Quali prompt usare?}
    \framesubtitle{Risposta troppo generica}

    \begin{itemize}
        \item Fare una domanda più precisa
        \begin{itemize}
            \item \textcolor{red}{\textit{Spiega dove mettere le posate}}
            \item \textcolor{DarkGreen}{\textit{Spiega dove mettere le posate nel contesto del bon ton, prestando particolare attenzione alla cultura asiatica (abbiamo 7 ospiti provenienti da ...)}}
        \end{itemize}
    \end{itemize}
\end{exampleframe}

\begin{exampleframe}
    \frametitle{Quali prompt usare?}
    \framesubtitle{Risposta troppo generica}

    \begin{itemize}
        \item Chiedere di risolvere solo una parte del problema
        \begin{itemize}
            \item \textcolor{red}{\textit{Spiega dove mettere le posate nel contesto del bon ton, prestando particolare attenzione alla cultura asiatica}}
            \item \textcolor{DarkGreen}{\textit{Spiega le differenze tra la cultura asiatica e quella italiana relativamente all'utilizzo delle posate nel bon ton}}
        \end{itemize}
    \end{itemize}
\end{exampleframe}

\begin{exampleframe}
    \frametitle{Quali prompt usare?}
    \framesubtitle{Risposta troppo generica}

    \begin{itemize}
        \item Fornire informazioni rilevanti sul materiale a nostra disposizione
        \begin{itemize}
            \item \textcolor{red}{\textit{Spiega dove mettere le posate nel contesto del bon ton, prestando particolare attenzione alla cultura asiatica}}
            \item \textcolor{DarkGreen}{\textit{Spiega dove mettere le posate nel contesto del bon ton, prestando particolare attenzione alla cultura asiatica. Il menu è il seguente: ...}}
        \end{itemize}
    \end{itemize}
\end{exampleframe}

\begin{exampleframe}
    \frametitle{Quali prompt usare?}
    \framesubtitle{Risposta troppo generica}

    \begin{itemize}
        \item Chiedere esempi
        \begin{itemize}
            \item \textit{Fornisci un esempio}
            \item \textit{Fornisci una lista di esempi}
            \item \textit{Spiega nuovamente quanto hai appena detto. Includi esempi per ogni passaggio}
        \end{itemize}
    \end{itemize}
\end{exampleframe}

\makesectionframe{Memoria tra le chat}

\begin{contentframe}
    \frametitle{Memoria tra le chat}

    \begin{itemize}
        \item Le AI generative si ricordano quanto detto \underline{solo} all'interno di una certa conversazione
        \begin{itemize}
            \item I messaggi precedenti influenzano le risposte successive
            \item Conversazioni diverse sono separate e non si influenzano a vicenda
        \end{itemize}
        
        \bigskip
        \item La stessa domanda all'interno di conversazioni diverse può dare risposte diverse!
    \end{itemize}
\end{contentframe}

\begin{contentframe}
    \frametitle{Dove scrivere un nuovo messaggio}

    \begin{itemize}
        \item Nella stessa conversazione
        \begin{itemize}
            \item Vogliamo fare una domanda collegata alla risposta precedente
        \end{itemize}

        \bigskip
        \item In una nuova conversazione
        \begin{itemize}
            \item Vogliamo fare una domanda non collegata ai messaggi precedenti
            \item \textit{Non abbiate paura di iniziare tante conversazioni, non c'è un limite!}
        \end{itemize}
    \end{itemize}
\end{contentframe}

\begin{contentframe}
    \frametitle{Risposte non corrette}

    Se la risposta non è quella che volevamo, possiamo:
    \bigskip
    
    \begin{itemize}
        \item Modificare il messaggio precedente
        \begin{itemize}
            \item Sentiamo di poter formulare meglio la domanda precedente
        \end{itemize}

        \bigskip
        \item Scrivere un altro messaggio
        \begin{itemize}
            \item Non sappiamo come formulare meglio la domanda
        \end{itemize}
    \end{itemize}
\end{contentframe}


\begin{exerciseframe}
    \frametitle{Esercizio}

    \begin{itemize}
        \item Far generare uno schema riassuntivo di quali prompt usare nelle varie situazioni, sulla base di quanto appena detto
    \end{itemize}
\end{exerciseframe}
