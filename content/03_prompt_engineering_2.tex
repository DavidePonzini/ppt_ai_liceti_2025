\makesectionframe{Prompt engineering}[Tecniche avanzate]

\begin{contentframe}
    \frametitle{Prompt Engineering}
    \framesubtitle{Tecniche avanzate}

    \begin{itemize}
        \item Diverse tecniche per ottenere risposte più accurate

        \begin{enumerate}
            \item Prompt Chaining
            \item Zero-shot Promping
            \item Few-shot Promping
            \item Directional Stimulus Prompting
            \item Chain-of-Thought Promping
            \item Meta Prompting
            \item Retrieval Augmented Generation (RAG)
        \end{enumerate}
    \end{itemize}
\end{contentframe}

\begin{contentframe}
    \frametitle{Prompt Chaining}

    \begin{block}{Prompt Chaining}
        Serie di prompt volti a ``ragionare'' assieme all'AI generativa, per guidarla a creare quello che vogliamo
    \end{block}

    \begin{itemize}
        \item In realtà è qualsiasi serie di messaggi nella stessa conversazione

        \bigskip
        \item \textbf{Consigliato per:} qualsiasi richiesta non banale
    \end{itemize}
\end{contentframe}

\begin{contentframe}
    \frametitle{Prompt Chaining}

    \begin{enumerate}
        \item Si parte con:
        \begin{itemize}
            \item Richiesta di spiegare i passaggi necessari alla risoluzione di un problema
            \item Richiesta di risolvere solo una parte del problema
        \end{itemize}

        \bigskip
        \item Si legge accuratamente la risposta
        \begin{itemize}
            \item Se non è esattamente quello che volevamo, si chiedono modifiche
        \end{itemize}
        
        \bigskip
        \item Si chiede di risolvere un'altra parte del problema
    \end{enumerate}
\end{contentframe}

\begin{exampleframe}
    \frametitle{Prompt Chaining}
    \framesubtitle{Esempi}

    \begin{itemize}
        \item \href{https://chatgpt.com/share/67618531-ffa0-8003-9b5c-a3901716ad68}{Pranzo a Genova}

        \bigskip
        \item \href{https://chatgpt.com/share/67617f2f-1fa8-8003-bbd3-50c23bcec90a}{Analisi follia Amleto}

        \bigskip
        \item Dialogo sul meteo
        \begin{itemize}
            \item \href{https://chatgpt.com/share/675b09fd-aeb8-8003-a291-b04cea1f17c8}{Singolo prompt \faSun}
            \item \href{https://chatgpt.com/share/675b098d-61b4-8003-881e-ba8f277e70b8}{Prompt Chaining \faWind~\faCloudShowersHeavy}
        \end{itemize}
    \end{itemize}
\end{exampleframe}

\begin{exerciseframe}
    \frametitle{Prompt Chaining}

    \begin{enumerate}
        \item Usando Prompt Chaining, inventare un menu per il cenone di Natale e preparare la lista della spesa. Assicurarsi che ogni piatto del menu sia di vostro gradimento

        \bigskip
        \item Usando Prompt Chaining, generare una breve relazione di qualità su un certo argomento
    \end{enumerate}
\end{exerciseframe}



\begin{contentframe}
    \frametitle{Zero-shot Prompting}

    \begin{itemize}
        \item Forma più semplice di prompting
        \item Si fa una domanda e si chiede una risposta

        \bigskip
        \item \textbf{Consigliato per:} qualsiasi richiesta
    \end{itemize}
\end{contentframe}

\begin{exampleframe}
    \frametitle{Zero-shot Prompting}
    \framesubtitle{Esempi}

    \begin{itemize}
        \item \href{https://chatgpt.com/share/675c4511-aa20-8003-ae94-2f77b3d3c6ad}{Classificazione sentimento testo}
        \item \href{https://chatgpt.com/share/675c46c2-8cec-8003-ba2f-434327e1fd49}{Riassunto testo}
        \item \href{https://chatgpt.com/share/675c4751-8c98-8003-aeb7-eeb3aeb805f2}{Python - pari o dispari}
    \end{itemize}
\end{exampleframe}

\begin{contentframe}
    \frametitle{Few-shot Prompting}

    \begin{itemize}
        \item Si formiscono uno o più esempi di domanda-risposta prima della domanda che ci interessa
        \item Gli esempi forniti influenzano il modo in cui viene formulata la risposta

        \bigskip
        \item \textbf{Consigliato per:} qualsiasi richiesta in cui Zero-shot Prompting non dà buoni risultati
    \end{itemize}
\end{contentframe}

\begin{exampleframe}
    \frametitle{Few-shot Prompting}
    \framesubtitle{Esempi}

    \begin{itemize}
        \item \href{https://chatgpt.com/share/67616888-2a94-8003-a6ff-d4654de2d19f}{Stile risposta - Zero-shot}
        \item \href{https://chatgpt.com/share/67617a94-bb78-8003-80e0-82d15c99033c}{Stile risposta - Few-shot}

        \bigskip
        \item \href{https://chatgpt.com/share/67852600-5a7c-8003-8e17-f9f33d8c9b18}{Sottintendere la domanda}
    \end{itemize}
\end{exampleframe}

\begin{exerciseframe}
    \frametitle{Zero-shot \& One-shot Prompting}
    \framesubtitle{Esercizi}

    \begin{enumerate}
        \item Effettuare qualche domanda usando Zero-shot Prompting e verificare la risposta
        \begin{itemize}
            \item Se non è di vostro gradimento, usare il Few-shot Prompting e osservare le differenze
        \end{itemize}

        \bigskip
        \item Fare una domanda senza formulare apertamente la richiesta \textit{(vedi slide precedente)}
    \end{enumerate}
\end{exerciseframe}



\begin{contentframe}
    \frametitle{Directional Stimulus Prompting}

    \begin{itemize}
        \item Si forniscono ``suggerimenti'' per indirizzare alla risposta

        \bigskip
        \item \textbf{Consigliato per:} generare risposte in cui sono enfatizzati alcuni concetti
    \end{itemize}
\end{contentframe}

\begin{exampleframe}
    \frametitle{Directional Stimulus Prompting}
    \framesubtitle{Esempio}

    \image*[.85]{dsp.jpg}

\end{exampleframe}

\begin{exerciseframe}
    \frametitle{Directional Stimulus Prompting}
    \framesubtitle{Esercizi}

    \begin{enumerate}
        \item Scrivere un racconto suggerendo alcuni concetti
        \begin{itemize}
            \item Esempio: \textit{avventura; aeroplano; amicizia; fame}
        \end{itemize}

        \bigskip
        \item Riassumere un testo assicurandosi di includere alcune informazioni specifiche \textit{(vedi slide precedente)}
    \end{enumerate}
\end{exerciseframe}


\begin{contentframe}
    \frametitle{Chain-of-Thought Prompting}

    \begin{itemize}
        \item Si chiede al modello di scrivere il procedimento che ha usato per risolvere il problema
        \begin{itemize}
            \item Il modello ora è ``obbligato'' a ragionare in quel modo
            \item Si riduce la probabilità che il modello ``tiri ad indovinare''
        \end{itemize}
        \item Per fare scrivere il ragionamento si possono usare le tecniche viste in precedenza

        \bigskip
        \item I modelli recenti lo fanno già in automatico in molti casi

        \bigskip
        \item \textbf{Consigliato per:} domande che richiedono di seguire un determinato un ragionamento
    \end{itemize}    
\end{contentframe}

\begin{exampleframe}
    \frametitle{Chain-of-Thought Prompting}
    \framesubtitle{Esempio}

    \image*{cot.jpg}
\end{exampleframe}

\begin{exerciseframe}
    \frametitle{Chain-of-Thought Prompting}
    \framesubtitle{Esercizi}

    \begin{enumerate}
        \item Porre alcune domande che richiedono ragionamento
        \begin{itemize}
            \item Se la risposta non contiene ragionamento, riprovare applicando Chain-of-Thought
            \item Se la risposta contiene ragionamento, provare con una domanda diversa
        \end{itemize}
    \end{enumerate}
    
\end{exerciseframe}

\begin{contentframe}
    \frametitle{Meta Prompting}

    \begin{itemize}
        \item Si descrive il formato in cui si vuole che il modello ragioni
        \item Non importa lo stile della risposta

        \bigskip
        \item Come per \textit{Chain-of-Thought}, il modello ragiona più attentamente sul procedimento per risolvere il problema

        \bigskip
        \item \textbf{Consigliato per:} domande che richiedono di ragionare attentamente prima di rispondere
    \end{itemize}
\end{contentframe}

\begin{exampleframe}
    \frametitle{Meta Prompting}
    \framesubtitle{Esempi}

    \begin{itemize}
        \item Calcolo del dominio di $\frac{\sqrt{-5x + 7}}{\ln(\sqrt{x})} + \cos^2\left(\frac{\pi}{4x}\right)$
        \begin{itemize}
            \item \href{https://chatgpt.com/share/6776910a-2990-8003-a158-e9337a55edf2}{Zero-shot Prompting}
            \item \href{https://chatgpt.com/share/6776a842-0c34-8003-adcb-9c45bbdbd9d8}{Meta Prompting}
            \item \href{https://www.wolframalpha.com/input?i2d=true&i=Divide\%5BSqrt\%5B-5x\%2B7\%5D\%2Cln\%5C\%2840\%29Sqrt\%5Bx\%5D\%5C\%2841\%29\%5D\%2BSquare\%5Bcos\%5C\%2840\%29Divide\%5B\%CF\%80\%2C4x\%5D\%5C\%2841\%29\%5Ddomain}{Soluzione corretta}
        \end{itemize}

        \bigskip
        \item Feedback su un elaborato scritto da uno studente
        \begin{itemize}
            \item \href{https://chatgpt.com/share/6776a962-9c08-8003-bdce-1787adc79515}{Zero-shot Prompting}
            \item \href{https://chatgpt.com/share/6776a77f-8288-8003-86ad-95b7b556fcf6}{Meta Prompting}
        \end{itemize}
    \end{itemize}
\end{exampleframe}

\begin{exerciseframe}
    \frametitle{Meta Prompting}
    \framesubtitle{Esercizio}

    \begin{itemize}
        \item Usando Chain-of-Thought o Meta Prompting, generare la risposta ad un problema che richiede ragionamento

        \bigskip
        \item Confrontare le risposte ottenute rispetto ad un prompt semplice
    \end{itemize}
\end{exerciseframe}


\begin{contentframe}
    \frametitle{Retrieval Augmented Generation (RAG)}

    \begin{itemize}
        \item Tecnica più complessa, supportata solo da alcuni modelli
        
        \bigskip
        \item Il modello ha accesso a uno o più documenti dell'utente
        \begin{itemize}
            \item Estrae informazioni pertinenti alla domanda
            \item Include automaticamente queste informazioni nella vostra domanda
            \item Genera la risposta usando queste informazioni
        \end{itemize}

        \bigskip
        \item \textbf{Consigliato per:} generare risposte attingendo a materiale fornito dall'utente
    \end{itemize}
\end{contentframe}

\begin{contentframe}
    \frametitle{``Retrieval Augmented Generation''}
    \framesubtitle{Versione \textit{fai-da-te}}

    \begin{itemize}
        \item Se sappiamo già quali sono le parti importanti di un testo, possiamo includerle direttamente noi nel prompt
        
        \bigskip
        \item Supportato da tutti i modelli, in quanto alla fine abbiamo un prompt normale

        \bigskip
        \item \textbf{Consigliato per:} generare risposte attingendo a materiale fornito dall'utente
    \end{itemize}
\end{contentframe}

\begin{exampleframe}
    \frametitle{Retrieval Augmented Generation}
    \framesubtitle{Esempi}

    \begin{itemize}
        \item \href{https://chatgpt.com/share/67769fb3-be30-8003-8ec1-74ecf8c9ad2d}{Zero-shot Prompt}
        \begin{itemize}
            \item Il modello non ha capito cosa intendevo con Meta Prompting ed ha generato un \textit{``prompt per generare prompt''}
        \end{itemize}
        \item \href{https://chatgpt.com/share/6776a11e-da24-8003-92ad-b574ebc8788a}{Aggiunta manuale del contesto} (\href{https://www.promptingguide.ai/techniques/meta-prompting}{fonte})
        \begin{itemize}
            \item Il modello ora ha capito cosa intendevo ed ha generato un esempio rilevante
        \end{itemize}
    \end{itemize}
\end{exampleframe}

\begin{exerciseframe}
    \frametitle{Retrieval Augmented Generation}
    \framesubtitle{Esercizi}

    \begin{enumerate}
        \item Fare una domanda e fornire un file contenente informazioni utili alla risposta

        \bigskip
        \item Chiedere di riassumere un documento allegato alla domanda

        \bigskip
        \item Chiedere di estrarre informazioni da un documento allegato alla domanda
    \end{enumerate}
\end{exerciseframe}