\makesectionframe{AI generativa nella didattica}[Rischi e opportunità]

\begin{contentframe}
    \frametitle{AI generativa nella didattica}

    \begin{itemize}
        \item La diffusione dell'AI generativa sta avendo un forte impatto in ambito didattico

        \bigskip
        \item Le conseguenze non sono ancora chiarissime
        \item Esistono sia rischi che opportunità
        \item Dipendono da come si usa lo strumento
    \end{itemize}
\end{contentframe}

\begin{contentframe}
    \frametitle{AI generativa nella didattica}

    \begin{itemize}
        \item Rischio più comune: affidarsi eccessivamente allo strumento
        \item Bisogna sempre ricordare che le risposte generate potrebbero:
        \begin{itemize}
            \item Essere evidentemente errate \textit{(molto raro, ma possibile)}
            \item Sembrare plausibili, ma contenere informazioni errate \textit{(poco frequente, ma avviene)}
            \item Essere prevalentemente corrette, ma con qualche errore o imprecisione difficile da notare \textit{(più frequente di quello che si pensa)}
            \item Essere corrette ma incomplete \textit{(mediamente frequente)}
            \item Essere corrette e complete \textit{(mediamente frequente)}
        \end{itemize}

        \bigskip
        \item Soluzione: essere consapevoli dei limiti dello strumento
    \end{itemize}
\end{contentframe}

\begin{exerciseframe}
    \frametitle{AI generativa per studenti}
    \framesubtitle{Esempio interattivo}

    \begin{enumerate}
        \item Svolgere il seguente compito \underline{usando} AI generativa
        \begin{itemize}
            \item Tempo a disposizione: 6 min
            \item Per ogni domanda rispondere scrivendo circa 10 righe
            \item \url{https://forms.gle/k3E9nKsYXwLWeowR7}
        \end{itemize}

        \pause
        \bigskip
        \item Rispondere al \href{https://play.blooket.com/host?id=678cef9809b01a90328de16a}{seguente quiz}
    \end{enumerate}
\end{exerciseframe}

\begin{contentframe}
    \frametitle{Rischi per studenti}

    \begin{columns}
        \col{.5}
        \begin{itemize}
            \item Riduzione apprendimento
            \item Distorsione apprendimento
            \item Riduzione sviluppo pensiero critico
            \item Dipendenza dallo strumento
        \end{itemize}
        
        \col{.5}
        \centering
        \image{student_sleeping.jpg}
    \end{columns}
\end{contentframe}

\begin{contentframe}
    \frametitle{Opportunità per studenti}

    \begin{columns}
        \col{.5}
        \begin{itemize}
            \item Supporto all'apprendimento
            \item Sviluppo pensiero critico
            \item Accesso a contenuti personalizzati
            \item Accesso a informazioni non reperibili su internet
            \item Aumento motivazione
            \item Disponibilità 24/7
        \end{itemize}
        
        \col{.5}
        \centering
        \image{student_smart.jpg}
    \end{columns}
\end{contentframe}

\begin{contentframe}
    \frametitle{Rischi per studenti}
    \framesubtitle{Riduzione apprendimento}

    \begin{itemize}
        \item Gli studenti potrebbero limitarsi a utilizzare l'output dell'AI senza comprendere a fondo il materiale

        \bigskip
        \item Non ``faticando'' per risolvere i problemi si impara di meno (o per niente)

        \bigskip
        \item \textit{Esempio: farsi fare i compiti e copiare meccanicamente le risposte}
    \end{itemize}
\end{contentframe}

\begin{contentframe}
    \frametitle{Rischi per studenti}
    \framesubtitle{Distorsione apprendimento}

    \begin{itemize}
        \item I bias e gli errori presenti nelle risposte potrebbero essere ``imparati'' dagli studenti

        \bigskip
        \item Le risposte errate potrebbero essere interpretate come corrette

        \bigskip
        \item \textit{Esempio: usare AI come dei motori di ricerca}
    \end{itemize}
\end{contentframe}

\begin{contentframe}
    \frametitle{Rischi per studenti}
    \framesubtitle{Riduzione sviluppo pensiero critico}

    \begin{itemize}
        \item Avere immediatamente una risposta a qualsiasi domanda porta a smettere di ragionare
        \item Particolarmente rilevante in soggetti più giovani (specialmente elementari/medie)

        \bigskip
        \item Senza pensiero critico non siamo in grado di distinguere le risposte corrette da quelle errate o imprecise

        \bigskip
        \item \textit{Esempio: assumere che tutte le risposte fornite dallo strumento siano corrette o che lo strumento sappia meglio di noi}
    \end{itemize}
\end{contentframe}

\begin{contentframe}
    \frametitle{Rischi per studenti}
    \framesubtitle{Dipendenza dallo strumento}

    \begin{itemize}
        \item Affidandosi sempre all'AI per ogni singolo compito, si diventa incapaci di svolgere attività senza

        \bigskip
        \item Non si sviluppano molte abilità utili per la vita
        
        \bigskip
        \item \textit{Esempio: affidarsi ad ogni minima difficoltà al supporto dell'AI}
    \end{itemize}
\end{contentframe}

\begin{contentframe}
    \frametitle{Opportunità per studenti}
    \framesubtitle{Supporto all'apprendimento}

    \begin{itemize}
        \item Un utilizzo consapevole può portare ad apprendere meglio concetti noti
        \item Un utilizzo consapevole può portare a scoprire nuovi concetti

        \bigskip
        \item \textit{Esempio: chiedere chiarimenti e spiegazioni o effettuare ricerche}
        \item Ricordarsi di verificare sempre le risposte ottenute!
    \end{itemize}
\end{contentframe}

\begin{contentframe}
    \frametitle{Opportunità per studenti}
    \framesubtitle{Sviluppo pensiero critico}

    \begin{itemize}
        \item Tutte le risposte fornite da AI andrebbero messe in dubbio, in quanto non sono verificate
        \item Questo processo porta allo sviluppo del pensiero critico

        \bigskip
        \item \textit{Esempio: ``In equitazione, il trotto è l'andatura più usata nel salto degli ostacoli''}
    \end{itemize}
\end{contentframe}

\begin{contentframe}
    \frametitle{Opportunità per studenti}
    \framesubtitle{Accesso a contenuti personalizzati}

    \begin{itemize}
        \item I contenuti creati da AI possono essere facilmente adattati alle necessità o agli interessi degli studenti
        \item Le lezioni possono diventare più coinvolgenti, usando argomenti verso cui gli studenti hanno interesse personale

        \bigskip
        \item \textit{Esempio: un problema di matematica può essere personalizzato in base agli sport frequentati dai singoli studenti}
    \end{itemize}
\end{contentframe}

\begin{exerciseframe}
    \frametitle{Accesso a contenuti personalizzati}
    \framesubtitle{Esercizio}

    \begin{itemize}
        \item Farsi spiegare un argomento con esempi pratici inerenti ad un vostro interesse
        \begin{itemize}
            \item \textit{``Spiega le frazioni con esempi pratici relativi alla cucina''}
            \item \textit{``Spiega la probabilità condizionata includendo esempi relativi ai videogiochi''}
            \item \textit{``Spiega le principali differenze tra la cultura occidentale e quella asiatica. Fornisci esempi relativi alle arti marziali''}
        \end{itemize}
    \end{itemize}
\end{exerciseframe}

\begin{contentframe}
    \frametitle{Opportunità per studenti}
    \framesubtitle{Accesso a informazioni non reperibili su internet}

    \begin{itemize}
        \item Nel caso su internet non si trovino risposte ad una certa domanda, lo strumento potrebbe generare comunque una risposta utile

        \bigskip
        \item Utile specialmente per scenari molto specifici o di nicchia
    \end{itemize}
\end{contentframe}

\begin{contentframe}
    \frametitle{Opportunità per studenti}
    \framesubtitle{Aumento motivazione}

    \begin{itemize}
        \item Utilizzare strumenti digitali può essere considerata attività ludica/ricreativa
        \begin{itemize}
            \item Se fatta nei modi e tempi appropriati, può aumentare la motivazione ad apprendere negli studenti
        \end{itemize}
        
        \bigskip
        \item Fare attenzione a circorscrivere bene l'attività ed ad inquadrarla in un contesto più grande

        \bigskip
        \item \textit{Esempio: pensare all'utilizzo di Kahoot. Si dedica un momento preciso della lezione ad usare AI generativa, con l'obiettivo di apprendere meglio i concetti trattati}
    \end{itemize}
\end{contentframe}

\begin{contentframe}
    \frametitle{Opportunità per studenti}
    \framesubtitle{Disponibilità 24/7}

    \begin{itemize}
        \item Lo strumento è disponibile in qualsiasi momento
        \bigskip
        \item I tempi di attesa sono ridotti a pochi secondi
    \end{itemize}
\end{contentframe}

\begin{contentframe}
    \frametitle{AI generativa per docenti}

    \begin{columns}
        \col{.5}
        \begin{itemize}
            \item AI generativa non è limitata agli studenti
            \item Anche i docenti possono usare questi strumenti nelle loro attività
    
            \bigskip
            \item Si può usare in tutti i modi descritti per gli studenti
            \begin{itemize}
                \item Sono presenti gli stessi rischi!
            \end{itemize}
            \item Si può usare anche a supporto delle lezioni
        \end{itemize}
        
        \col{.5}
        \centering
        \image{teacher_student.jpg}
    \end{columns}
\end{contentframe}

\begin{exerciseframe}
    \frametitle{AI generativa per docenti}
    \framesubtitle{Esempio interattivo}

    Formulando esclusivamente prompt semplici:
    \begin{enumerate}
        \item Far generare esercizi per una verifica
        \begin{itemize}
            \item \textit{``Prepara degli esercizi di ripasso per una verifica sui Promessi Sposi''}
        \end{itemize}

        \pause
        \bigskip
        \item Correggere esercizi svolti da studenti ed assegnare un voto
        \begin{itemize}
            \item \textit{``Correggi il seguente compito: ...''}
            \item \textit{``Fornisci un voto in decimi per il seguente compito: ...''}
        \end{itemize}
    \end{enumerate}
\end{exerciseframe}

\begin{contentframe}
    \frametitle{Rischi per docenti}

    \begin{columns}
        \col{.5}
        \begin{itemize}
            \item Generazione di materiali o esercizi banali/superficiali
            \item Giudizi non coerenti o pertinenti
        \end{itemize}
    
        \col{.5}
        \centering
        \image{teacher_bad.jpg}
    \end{columns}
\end{contentframe}

\begin{contentframe}
    \frametitle{Opportunità per docenti}

    \begin{columns}
        \col{.5}
        \begin{itemize}
            \item Risparmio di tempo \textit{(soprattutto per operazioni meccaniche)}
            \item Analisi più approfondite rispetto a strumenti tradizionali
            \item Generazione materiale personalizzato
            \item Generazione feedback personalizzato
            \item Revisione materiale creato
        \end{itemize}
        
        \col{.5}
        \centering
        \image{teacher_good.jpg}
    \end{columns}
\end{contentframe}

\begin{contentframe}
    \frametitle{Rischi per docenti}
    \framesubtitle{Generazione di materiali o esercizi banali/superficiali}

    \begin{itemize}
        \item Gli esercizi generati tramite richieste semplici tendono:
        \begin{itemize}
            \item Ad avere difficoltà medio-bassa
            \item A non essere coerenti col programma
            \item A non ricoprire tutti gli aspetti rilevanti
        \end{itemize}

        \bigskip
        \item Con richieste più elaborate si possono ottenere buoni risultati
        
        \bigskip
        \item \textit{Esempio: ``genera degli esercizi sui verbi in inglese''}
    \end{itemize}
\end{contentframe}

\begin{contentframe}
    \frametitle{Rischi per docenti}
    \framesubtitle{Giudizi non coerenti o pertinenti}

    \begin{itemize}
        \item Correzioni e valutazioni potrebbero:
        \begin{itemize}
            \item Non essere esaustive
            \item Non essere coerenti col programma
            \item Non essere coerenti tra di loro
        \end{itemize}

        \bigskip
        \item Richieste elaborate mitigano questo problema ma non lo eliminano
    \end{itemize}
\end{contentframe}

\begin{contentframe}
    \frametitle{Opportunità per docenti}
    \framesubtitle{Risparmio di tempo}

    \begin{itemize}
        \item È possibile generare molto velocemente contenuti banali

        \bigskip
        \item Molto utile per la burocrazia
        \item Utile per generare una traccia da adattare successivamente
    \end{itemize}
\end{contentframe}

\begin{contentframe}
    \frametitle{Opportunità per docenti}
    \framesubtitle{Analisi più approfondite rispetto a strumenti tradizionali}

    \begin{itemize}
        \item È possibile analizzare molto velocemente anche aspetti complessi di un elaborato

        \bigskip
        \item Attenzione: soprattutto con prompt semplici, non è garantito che le analisi siano accurate o comprensive

        \bigskip
        \item \textit{Esempio: ``descrivi la padronanza del linguaggio tecnico in questo elaborato''}
    \end{itemize}
\end{contentframe}

\begin{contentframe}
    \frametitle{Opportunità per docenti}
    \framesubtitle{Generazione materiale personalizzato}

    \begin{itemize}
        \item Si possono generare velocemente materiali personalizzati in base alle esigenze
        \item Come sempre, è necessario controllare manualmente quanto prodotto

        \bigskip
        \item \textit{Esempio: ``genera uno schema riassuntivo dei principali tipi di proposizione''} 
    \end{itemize}
\end{contentframe}

\begin{contentframe}
    \frametitle{Opportunità per docenti}
    \framesubtitle{Generazione feedback personalizzato}

    \begin{itemize}
        \item Si possono generare velocemente feedback personalizzati in base agli elaborati prodotti ed alle esigenze del docente

        \bigskip
        \item \textit{Esempio: \href{https://chatgpt.com/share/6776a77f-8288-8003-86ad-95b7b556fcf6}{Feedback elaborato}}
    \end{itemize}
\end{contentframe}

\begin{contentframe}
    \frametitle{Opportunità per docenti}
    \framesubtitle{Revisione materiale creato}

    \begin{itemize}
        \item Si può usare l'AI per migliorare il materiale creato:
        \begin{itemize}
            \item Chiedendo un parere/giudizio
            \item Chiedendo di suggerire miglioramenti
        \end{itemize}

        \bigskip
        \item Si può chiedere di risolvere gli esercizi e vedere se ha capito correttamente la consegna
    \end{itemize}
\end{contentframe}

\begin{exerciseframe}
    \frametitle{AI generativa per docenti}
    \framesubtitle{Esempio interattivo}

    Formulando in maniera elaborata i prompt:
    \begin{enumerate}
        \item Far generare esercizi in preparazione ad una verifica
        \begin{itemize}
            \item Esempio: \href{https://chatgpt.com/share/678ba114-e39c-8003-9f3e-cba11d1fea00}{Promessi Sposi}
        \end{itemize}

        \bigskip
        \item Far generare esercizi da includere all'interno di una verifica
        
        \bigskip
        \item Correggere esercizi svolti da studenti ed assegnare un voto
        \begin{itemize}
            \item \textit{``Correggi il seguente compito, usando i seguenti criteri ed applicando la seguente griglia di valutazione: ...''}
        \end{itemize}
    \end{enumerate}
\end{exerciseframe}

