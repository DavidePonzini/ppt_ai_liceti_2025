\makesectionframe{Progettazione didattica per AI generativa}

\begin{contentframe}
    \frametitle{Utilizzi didattici AI generativa}

    Molti utilizzi possibili, tra cui:
    
    \begin{columns}
        \col{.5}
        \begin{itemize}
            \item Esplorazione
            \item Spiegazioni
            \item Generazione esercizi
            \item Risoluzione esercizi
        \end{itemize}
        
        \col{.5}
        \begin{itemize}
            \item Supporto a studenti con difficoltà
            \item Preparazione materiale per le lezioni
            \item Correzione esercizi
            \item Valutazione esercizi
        \end{itemize}
    \end{columns}
\end{contentframe}

\begin{contentframe}
    \frametitle{Esplorazione}

    \begin{itemize}
        \item Tramite AI generativa è possibile esplorare ed imparare concetti sconosciuti
        \item \underline{È fondamentale essere guidati dalla curiosità}
        \begin{itemize}
            \item La prima risposta non sarà mai esaustiva, bisogna continuare a chiedere domande
            \item L'obiettivo non deve essere fare un certo numero di domande, ma tentare di capire quanto più possibile
        \end{itemize}

        \bigskip
        \item \textit{Vediamo un esempio...}
    \end{itemize}
    
\end{contentframe}

\begin{exerciseframe}
    \frametitle{Esplorazione}
    \framesubtitle{Esercizio}

    \begin{itemize}
        \item Scoprire un argomento nuovo con cui non siete familiari
        \begin{itemize}
            \item Tempo a disposizione: almeno 10 min
            \item Argomento:\pause
                \begin{CJK}{UTF8}{min}
                    Kana (ひらがな・カタカナ)
                \end{CJK}
        \end{itemize}

        \pause
        \bigskip
        \item Svolgere un \href{https://play.blooket.com/host?id=67855de45130aefd0c2e869f}{quiz} sui concetti appena appresi
    \end{itemize}
\end{exerciseframe}

\begin{contentframe}
    \frametitle{Spiegazioni}

    \begin{itemize}
        \item Tramite AI generativa si possono ottenere spiegazioni personalizzate su qualsiasi argomento

        \bigskip
        \item È consigliata la presenza del docente per:
        \begin{itemize}
            \item Guidare gli studenti a formulare bene le domande
            \item Valutare la correttezza delle risposte
            \item Insegnare agli studenti ad essere autonomi
        \end{itemize}
    \end{itemize}
\end{contentframe}

\begin{contentframe}
    \frametitle{Generazione esercizi}

    \begin{itemize}
        \item Tramite AI generativa si possono generare esercizi personalizzati per allenamento o ripasso

        \bigskip
        \item È consigliata la presenza del docente per:
        \begin{itemize}
            \item Guidare gli studenti a formulare bene le richieste
            \item Valutare che le domande generate siano coerenti con gli obiettivi didattici
            \item Valutare (anche informalmente) le risposte fornite dagli studenti
        \end{itemize}
    \end{itemize}
\end{contentframe}

\begin{contentframe}
    \frametitle{Risoluzione esercizi}

    \begin{itemize}
        \item Tramite AI generativa si possono risolvere esercizi
        \begin{itemize}
            \item Utile per fornire un confronto con le risposte date dagli studenti
            \item Utile per quando gli studenti sono bloccati su un esercizio
        \end{itemize}

        \bigskip
        \item È consigliata la presenza del docente per:
        \begin{itemize}
            \item Sensibilizzare gli studenti sui rischi di affidarsi troppo a questi strumenti
            \item Guidare gli studenti a formulare bene le richieste
            \item Valutare che le risposte generate non contengano errori o imprecisioni
            \item Valutare che le risposte generate siano coerenti con gli obiettivi didattici
        \end{itemize}
    \end{itemize}
\end{contentframe}

\begin{contentframe}
    \frametitle{Studenti con difficoltà}

    \begin{itemize}
        \item Tramite AI generativa si possono aiutare studenti con:
        \begin{itemize}
            \item Difficoltà linguistiche
            \item Difficoltà cognitive
        \end{itemize}

        \bigskip
        \item È consigliata la presenza del docente per:
        \begin{itemize}
            \item Guidare gli studenti a formulare bene le richieste
            \item Assicurarsi che le risposte generate siano utili per gli studenti
        \end{itemize}
    \end{itemize}
\end{contentframe}

\begin{contentframe}
    \frametitle{Preparazione materiale}

    \begin{itemize}
        \item I docenti possono usare AI generativa per preparare o adattare materiale utile alle lezioni

        \bigskip
        \item È consigliato fare attenzione a:
        \begin{itemize}
            \item Controllare la correttezza dei contenuti generati
            \item Controllare la coerenza dei contenuti generati con gli obiettivi didattici
        \end{itemize}
    \end{itemize}
\end{contentframe}

\begin{contentframe}
    \frametitle{Preparazione esercizi}

    \begin{itemize}
        \item I docenti possono usare AI generativa per preparare o adattare esercizi o verifiche

        \bigskip
        \item È consigliato fare attenzione a:
        \begin{itemize}
            \item Controllare la correttezza dei contenuti generati
            \item Controllare la coerenza dei contenuti generati con gli obiettivi didattici
            \item Assicurarsi che gli esercizi proposti siano sufficientemente esaustivi
        \end{itemize}
    \end{itemize}
\end{contentframe}

\begin{contentframe}
    \frametitle{Correzione esercizi}

    \begin{itemize}
        \item Tramite AI generativa si possono correggere esercizi o fornire feedback personalizzati

        \bigskip
        \item È consigliata la presenza del docente per:
        \begin{itemize}
            \item Guidare gli studenti a formulare bene le richieste
            \item Insegnare agli studenti ad usare consapevolmente i feedback forniti
            \item Valutare che le correzioni siano corrette e coerenti con gli obiettivi didattici
        \end{itemize}
    \end{itemize}
\end{contentframe}

\makesectionframe{Progettazione didattica con AI generativa}

\begin{exerciseframe}
    \frametitle{Progettazione didattica con AI generativa}
    \framesubtitle{Attività di gruppo}

    Progettiamo una lezione in cui gli studenti dovranno usare AI generativa
    \begin{enumerate}
        \item Scegliere un argomento ed un tipo di pubblico
        \item Decidere gli obiettivi didattici
        \item Decidere gli argomenti da trattare
        \item Decidere come gli studenti useranno AI generativa
        \item Preparare esercizi o una prova finale
        \item Preparare una griglia di valutazione
        \item Preparare i materiali per la lezione
    \end{enumerate}
\end{exerciseframe}