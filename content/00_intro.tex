% =============================================================
%   PRESENTAZIONE CORSO
% =============================================================

\makesectionframe{Presentazione corso}

\begin{contentframe}
    \frametitle{Obiettivi del corso}

    \begin{itemize}
        \item Conoscere cosa sono le intelligenze artificiali generative ed i rischi associati ad esse
        
        \bigskip
        \item Imparare a formulare richieste efficaci

        \bigskip
        \item Comprendere come usare questi strumenti nella didattica e nell'apprendimento
    \end{itemize}
\end{contentframe}

\begin{contentframe}
    \frametitle{Contenuti e struttura del corso}

    \begin{enumerate}
        \item Intelligenza artificiale e intelligenza artificiale generativa
        \item Prompt engineering base
        \item Prompt engineering avanzato
        \item Applicazioni in didattica per studenti e docenti
        \item Progettazione didattica per AI generativa
    \end{enumerate}
\end{contentframe}

% \begin{contentframe}
%     \frametitle{Strumenti richiesti}

%     \begin{columns}
%         \col{.5}
%         Per lavorare in locale:
%         \begin{itemize}    
%             \item Interprete Python\footnote[frame]{\url{https://www.python.org/downloads/}}
%             \item Editor di testo (VS Code\footnote[frame]{\url{https://code.visualstudio.com/download}})
%         \end{itemize}

%         \col{.5}
%         Per lavorare online:
%         \begin{itemize}
%             \item Account Replit\footnote[frame]{\url{https://replit.com/}}
%             \bigskip
%         \end{itemize}
%     \end{columns}
% \end{contentframe}

\begin{contentframe}
    \frametitle{Conosciamoci!}
    
    Quale immagine vi rappresenta di più?

    \bigskip
    \begin{columns}
        \col{.3}
        \centering
        \image{intro1.jpg}
        
        \col{.3}
        \centering
        \image{intro2.jpg}
        
        \col{.3}
        \centering
        \image{intro3.jpg}
    \end{columns}
\end{contentframe}
